%% Packages élémentaires %%
\usepackage[utf8]{inputenc}
\usepackage{mathpazo,etoolbox, graphicx, wrapfig, pbox, fancybox, hyperref, appendix, geometry, amsmath, amssymb, tikz, pgfplots, calc, enumitem, colortbl, subfiles}
\usepackage{forest} % Pour une arborescence de fichiers
%% supression des alineas
\setlength{\parindent}{0pt}

%% Ajustement des marges
\geometry{hmargin=2.4cm, vmargin = 2.1cm}

%% Changement des "---" en "--" dans itemize
\setlist[itemize]{itemsep=10pt, label={--}}

%% Couleurs %%
\usepackage{xcolor}
\definecolor{bleu}{RGB}{14, 68, 175}
\definecolor{BGbleu}{RGB}{222, 233, 255 }
\definecolor{BGorange}{RGB}{255, 216, 154}
\definecolor{rouge}{RGB}{201, 0, 0}
\definecolor{vert}{RGB}{14, 137, 0}
\definecolor{BGgris}{RGB}{222,230,230}

% Couleurs additionnelles jolies
\definecolor{vertPale}{RGB}{227, 255, 233}
\definecolor{bleuPale}{RGB}{232, 238, 255}
\definecolor{jaunePale}{RGB}{253, 255, 189}
\definecolor{orangePale}{RGB}{255, 223, 189}
\definecolor{rougePale}{RGB}{255, 217, 217}
\definecolor{mauvePale}{RGB}{249, 222, 255}

\newcommand\rouge[1] {{\color{rouge}{#1}}}
\newcommand\bleu[1] {{\color{bleu}{#1}}}
\newcommand\term[1] {\textbf{{\color{bleu}{#1}}}}
\newcommand\green[1]{{\color{vert}{#1}}}
\newcommand\folder[1]{\green{{#1}}}
\newcommand\subfolder[1]{\bleu{{#1}}}

%% Checkbox : https://tex.stackexchange.com/questions/16000/creating-boxed-check-mark
\newcommand{\checked}{$\mbox{\ooalign{$\checkmark$\cr\hidewidth$\square$\hidewidth\cr}}$}
\newcommand{\unchecked}{$\square$}
%% Cadres %%

\newcommand{\custombox}[4]{
    \fcolorbox{#4}{#1}{\parbox{#2\linewidth}{ 
    #3
    }}
}

\newcommand{\presentationBox}[3]{
    \custombox{#2}{0.45}{
        \begin{center}
            #1
        \end{center}
        \begin{small}
            \textit{#3}
        \end{small}
    }{white}    
}
\setlength{\fboxsep}{1em} % espace entre le bord d'une boite et le texte dedans

\newcommand\bbm[1]{
\begin{center}
    \fcolorbox{black}{BGbleu}{\parbox{\linewidth}{ 
    #1
    }}
\end{center}
}


\newcommand\bo[1]{
\begin{center}
\fcolorbox{black}{BGorange}{\parbox{\textwidth}{ 
#1
}}
\end{center}}

\newcommand\bb[1]{
\begin{center}
\fcolorbox{black}{BGbleu}{\parbox{\textwidth}{ 
\begin{Large}
\begin{center}
#1
\end{center}
\end{Large}
}}
\end{center}}


\newcommand\boite[1]{
\begin{center}
\fbox{\parbox{\textwidth}{ \begin{center}
\begin{Large}
#1
\end{Large}
\end{center}}}
\end{center}}

\newcommand\aparte[1]{
\begin{center}
\fcolorbox{white}{BGgris}{\parbox{\linewidth}{ \textit{A parte} \\
#1 }}
\end{center}}
\newcommand\bg[2]{
\begin{center}
\fcolorbox{white}{BGgris}{\parbox{\linewidth}{ \textit{#1} \\
#2 }}
\end{center}}

\newcommand\exemple[1]{
\begin{center}
\fcolorbox{white}{BGgris}{\parbox{\linewidth}{ \textit{Exemple} \\
#1 }}
\end{center}}

%% Commandes %%
\newcommand\imp[1]{\underline{\textbf{#1}}}
\newcommand\eq[1]{\begin{large}
\begin{align*}
#1
\end{align*}
\end{large}}
%% Commandes fantaisistes (cf. Internet) %%
\renewcommand{\parallel}{ \mathbin{\!/\mkern-5mu/\!} }
\newcommand{\q}[1]{{%
\font\larm = larm1000%
\larm%
\char 190}{ \textit{#1} }{%
\font\larm = larm1000%
\larm%
\char 191}}

%% Wrapping %%
\newcommand\wrap[4]{\begin{wrapfigure}[#1]{#2}{#3\textwidth}
#4
\end{wrapfigure}}



%% Code %%
\usepackage{listings}
\definecolor{codegreen}{rgb}{0,0.6,0}
\definecolor{codegray}{rgb}{0.5,0.5,0.5}
\definecolor{codepurple}{rgb}{0.58,0,0.82}
\definecolor{backcolour}{RGB}{242,242,242}
\definecolor{codeorange}{RGB}{255,140,0}

\newcommand{\code}[1]{\texttt{{\color{codepurple}{#1}}}}
\newcommand{\codep}[1]{\texttt{{\color{codegreen}{#1}}}}

\lstdefinestyle{mystyle}{
    backgroundcolor=\color{backcolour},   
    commentstyle=\color{bleu},
    keywordstyle=\color{codeorange},
    numberstyle=\tiny\color{codegray},
    stringstyle=\color{codepurple},
    basicstyle=\ttfamily\footnotesize,
    breakatwhitespace=false,         
    breaklines=true,                 
    captionpos=b,                    
    keepspaces=true,                 
    numbers=left,                    
    numbersep=5pt,                  
    showspaces=false,                
    showstringspaces=false,
    showtabs=false,                  
    tabsize=2
}

\lstset{style=mystyle}
\lstset{language=C}

\lstdefinestyle{cppstyle}{
basicstyle=\footnotesize\sffamily\color{black},
commentstyle=\color{mygray},
frame=single,
numbers=left,
numbersep=5pt,
numberstyle=\tiny\color{mygray},
keywordstyle=\color{mygreen},
showspaces=false,
showstringspaces=false,
stringstyle=\color{myorange},
tabsize=2
}


%% Leftbar (merci M. Albrecht-Marc)
\usepackage{framed}

\renewenvironment{leftbar}[1][\hsize]
{%
     \def\FrameCommand
     {%
         {\color{black}\vrule width 3pt}%
         \hspace{7pt}%must no space.
        % \fboxsep=\FrameSep\colorbox{yellow}%
     }%
     \MakeFramed{\hsize#1\advance\hsize-\width\FrameRestore}%
}
{\endMakeFramed}
\newenvironment{blueleftbar}[1][\hsize]
{%
     \def\FrameCommand
     {%
         {\color{blue}\vrule width 3pt}%
         \hspace{7pt}%must no space.
        % \fboxsep=\FrameSep\colorbox{yellow}%
     }%
     \MakeFramed{\hsize#1\advance\hsize-\width\FrameRestore}%
}
{\endMakeFramed}
